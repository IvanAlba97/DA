\begin{lstlisting}
(La clase 'Cell' y la función 'selection' se encuentran implementadas en el ejercicio 3)
void DEF_LIB_EXPORTED placeDefenses(bool** freeCells, int nCellsWidth, int nCellsHeight, float mapWidth, float mapHeight
              , std::list<Object*> obstacles, std::list<Defense*> defenses) {
    float cellWidth = mapWidth / nCellsWidth;
    float cellHeight = mapHeight / nCellsHeight; 
    float values[nCellsHeight][nCellsWidth];    // Matriz que almacena los valores de cada celda
    List<Cell> cellList;
    // Rellenamos la matriz de valores para colocar el resto de defensas
    for(int i = 0; i < nCellsHeight; i++) {
        for(int j = 0; j < nCellsWidth; j++) {
            values[i][j] = cellValueRest(i, j, freeCells, nCellsWidth, nCellsHeight, mapWidth, mapHeight, obstacles, defenses);
        }
    }
    // Vaciamos la lista
    while(!cellList.empty()) { cellList.pop_front(); }
    // Agrupamos las celdas en la lista anterior
    itCells = cellList.begin();
    for(int i = 0; i < nCellsHeight; i++) {
        for(int j = 0; j < nCellsWidth; j++) {
            Cell c(i, j, values[i][j]);
            cellList.insert(itCells, c);
            itCells++;
        }
    }
    // Ordenamos la lista de menor a mayor valor
    cellList.sort([](Cell& c1, Cell& c2) { return c1.value() < c2.value(); });
    // Colocamos el resto de defensas
    itDefense = defenses.begin();
    itDefense++;    // Empezamos a colocar la segunda defensa, puesto que la primera es el centro de extracción
    while(itDefense != defenses.end()) {    // Mientras haya defensas sin colocar...
        placed = false;
        while(!placed) {
            // Seleccionamos la celda más prometedora, que es el primer elemento de cellList
            Cell promisingCell = selection(cellList);  // promisingCell guarda la celda más prometedora
            if(factibility(*itDefense, promisingCell.row(), promisingCell.col(), obstacles, defenses, cellWidth, cellHeight, mapWidth, mapHeight)) {
                // Colocamos la defensa en el centro de la celda más prometedora
                (*itDefense)->position = cellCenterToPosition(promisingCell.row(), promisingCell.col(), cellWidth, cellHeight);
                placed = true;
            }
            // Sacamos de la lista la celda procesada
            cellList.pop_front();
        }
        itDefense++;
    }
}
\end{lstlisting}