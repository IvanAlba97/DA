\documentclass[]{article}

\usepackage[left=2.00cm, right=2.00cm, top=2.00cm, bottom=2.00cm]{geometry}
\usepackage[spanish,es-noshorthands]{babel}
\usepackage[utf8]{inputenc} % para tildes y ñ
\usepackage{graphicx} % para las figuras
\usepackage{xcolor}
\usepackage{listings} % para el código fuente en c++

\lstdefinestyle{customc}{
  belowcaptionskip=1\baselineskip,
  breaklines=true,
  frame=single,
  xleftmargin=\parindent,
  language=C++,
  showstringspaces=false,
  basicstyle=\footnotesize\ttfamily,
  keywordstyle=\bfseries\color{green!40!black},
  commentstyle=\itshape\color{gray!40!gray},
  identifierstyle=\color{black},
  stringstyle=\color{orange},
}
\lstset{style=customc}


%opening
\title{Práctica 3. Divide y vencerás}
\author{Nombre Apellido1 Apellido2 \\ % mantenga las dos barras al final de la línea y este comentario
correo@servidor.com \\ % mantenga las dos barras al final de la línea y este comentario
Teléfono: xxxxxxxx \\ % mantenga las dos barras al final de la linea y este comentario
NIF: xxxxxxxxm \\ % mantenga las dos barras al final de la línea y este comentario
}


\begin{document}

\maketitle

%\begin{abstract}
%\end{abstract}

% Ejemplo de ecuación a trozos
%
%$f(i,j)=\left\{ 
%  \begin{array}{lcr}
%      i + j & si & i < j \\ % caso 1
%      i + 7 & si & i = 1 \\ % caso 2
%      2 & si & i \geq j     % caso 3
%  \end{array}
%\right.$

\begin{enumerate}
\item Describa las estructuras de datos utilizados en cada caso para la representación del terreno de batalla. 

La función dedicada a establecer un determinado valor a cada una de las celdas del terreno de batalla la hemos llamado cellValue().

\begin{figure}
\centering
\includegraphics[width=0.7\linewidth]{./defenseValueCellsHead}
\caption{Estrategia devoradora para la mina}
\label{fig:defenseValueCellsHead}
\end{figure}

\item Implemente su propia versión del algoritmo de ordenación por fusión. Muestre a continuación el código fuente relevante. 

Escriba aquí su respuesta al ejercicio 2.



\item Implemente su propia versión del algoritmo de ordenación rápida. Muestre a continuación el código fuente relevante. 

void insertionSort(std::vector<Cell>& cellVector, int i, int j) {
    Cell cell;
    for (int l = i + 1; l <= j; l++) {
        cell = cellVector[l];
        int p = l - 1;
        while (p >= i && cellVector[p].getValue() < cell.getValue()) {
            cellVector[p+1] = cellVector[p];
            p--;
        }
        cellVector[p+1] = cell;
    }
}

int pivote(std::vector<Cell>& cellVector, int i, int j) {
    int p = i;
    Cell x = cellVector[i];
    for(int k = i + 1; k <= j; k++) {
        if(cellVector[k].getValue() <= x.getValue()) {
            p++;
            Cell aux = cellVector[p];
            cellVector[p] = cellVector[k];
            cellVector[k] = aux;
        }
    }
    cellVector[i] = cellVector[p];
    cellVector[p] = x;
    return p;
}

void fastSort(std::vector<Cell>& cellVector, int i, int j) {
    int n = j - i + 1;
    int umbral = 2;
    if(n <= umbral) {
        insertionSort(cellVector, i, j);
    } else {
        int p = pivote(cellVector, i, j);
        fastSort(cellVector, i, p - 1);
        fastSort(cellVector, p + 1, j);
    }
}

\item Realice pruebas de caja negra para asegurar el correcto funcionamiento de los algoritmos de ordenación implementados en los ejercicios anteriores. Detalle a continuación el código relevante.

\begin{lstlisting}
void selectIds(float** tsp, unsigned int ases, std::list<Aux> myDefenses, std::list<int> &selectedIDs) {
    std::list<Aux>::iterator it = myDefenses.end()--;   // end() devuelve la siguiente posición a la última, por eso el --
    int j = ases;
    for(int i = myDefenses.size()-1; i > 0; i--, it--) {
        if(tsp[i][j] != tsp[i-1][j]) {
            selectedIDs.push_back(it->getId());
            //std::cout<<it->getId()<<std::endl;
            j = j - it->getCost();
        }
    }
    if(tsp[0][j] != 0) {
        selectedIDs.push_back(myDefenses.begin()->getId());
    }
}
\end{lstlisting}

\item Analice de forma teórica la complejidad de las diferentes versiones del algoritmo de colocación de defensas en función de la estructura de representación del terreno de batalla elegida. Comente a continuación los resultados. Suponga un terreno de batalla cuadrado en todos los casos. 

Escriba aquí su respuesta al ejercicio 5.

\item Incluya a continuación una gráfica con los resultados obtenidos. Utilice un esquema indirecto de medida (considere un error absoluto de valor 0.01 y un error relativo de valor 0.001). Es recomendable que diseñe y utilice su propio código para la medición de tiempos en lugar de usar la opción \emph{-time-placeDefenses3} del simulador. Considere en su análisis los planetas con códigos 1500, 2500, 3500,..., 10500, al menos. Puede incluir en su análisis otros planetas que considere oportunos para justificar los resultados. Muestre a continuación el código relevante utilizado para la toma de tiempos y la realización de la gráfica.

\begin{lstlisting}
// sustituya este codigo por su respuesta
void placeDefenses(...) {

    List<Defense*>::iterator currentDefense = defenses.begin();
    while(currentDefense != defenses.end() && maxAttemps > 0) {

        (*currentDefense)->position.x = ((int)(_RAND2(nCellsWidth))) * cellWidth + cellWidth * 0.5f;
        ...
        ++currentDefense;
    }
}
\end{lstlisting}

\end{enumerate}

Todo el material incluido en esta memoria y en los ficheros asociados es de mi autoría o ha sido facilitado por los profesores de la asignatura. Haciendo entrega de este documento confirmo que he leído la normativa de la asignatura, incluido el punto que respecta al uso de material no original.

\end{document}
